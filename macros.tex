% --- Define gitbash language ---
\lstdefinelanguage{gitbash}{
    keywords={commit,push, status, add, diff, rm, switch, stash, pop, clone, init, branch},
    keywordstyle=\color{orange}\bfseries,
    %keywords not needed as long as preloading works
    %morekeywords=[2]{\$,git},
    %keywordstyle=[2]\color{green}\bfseries,
    morekeywords=[3]{-a,-c,-m,-s,--short,-M,-u},
    keywordstyle=[3]\color{green}\bfseries,
    morekeywords=[4]{<file>, <file_from>,<file_to>, <branch>, <new_branch>, <url>},
    keywordstyle=[4]\itshape,
    morekeywords=[5]{mkdir},
    keywordstyle=[5]\color{red}\bfseries,
    alsoletter={-,<,>},
}

\lstset{
    basicstyle=\ttfamily,
    language=gitbash,
    columns=fullflexible
}

% --- tcolorbox styles ---
\tcbset{
splitBox/.style={
    listing engine=minted,
    minted language=bash,
    listing side text,
    sidebyside,
    boxrule=1pt,
    skin=bicolor,
    colback=gray!10,
    colbacklower=white,
    valign=center,
    top=2pt, bottom=2pt, left=2pt, right=2pt
}}

\tcbset{
splitBoxCode/.style={
    sidebyside,
    sidebyside gap=5mm,
    listing engine=minted,
    minted language=bash,
    listing only,
    boxrule=1pt,
    skin=bicolor,
    colback=gray!10,
    colbacklower=white,
    valign=center,
    top=2pt, bottom=2pt, left=2pt, right=2pt
}}

\lstdefinestyle{python-clean}{
    language=Python,
    basicstyle=\ttfamily\small,
    frame=single,
    rulecolor=\color{grayframe},
    breaklines=true,
    captionpos=b,
    numbers=left,
    numberstyle=\tiny\color{gray},
    keywordstyle=\color{darkblue}\bfseries,
    commentstyle=\color{darkgreen}\itshape,
    stringstyle=\color{darkred},
    tabsize=4,
    showstringspaces=false,
    emph={self,True,False,None}, % extra highlighting
    emphstyle=\color{blue}
}

% --- tcblisting styles ---
\newtcblisting{gitBashBox} {
    colback=black,
    colupper=white,
    listing engine = listings,
    listing only,
    listing options={
        style=tcblatex,
        language=gitbash},
    every listing line={
    \textcolor{white}{\small\ttfamily\bfseries 
    \$git }},
    sharp corners
}

% --- Create inline code boxes ---
%use: ... \gitinline{} ...
\newcommand{\gitinline}[1]{%
  \colorbox{black}{\textcolor{white}{\$\lstinline!#1!}}%
}

\newcommand{\cmdinline}[1]{%
  \colorbox{black}{\textcolor{white}{\lstinline!#1!}}%
}

\newcommand{\nln}{\newline\noindent}


%-------------------------
%   Macros for Apendix A
%-------------------------
\definecolor{mygrey}{HTML}{ebebeb}
\definecolor{myblue}{HTML}{21436c}


\tcbset{
innerBox/.style={
  colback=white,
  colframe=white,
  bicolor,
  sharp corners,
  colbacklower=mygrey,   % gray lower background  % title uses Inter
  fontupper= \small, %\interfont,             % upper content uses Inter
  fontlower= \large,%\jetbrains
}}

\tcbset{
titleBox/.style={
  enhanced,  
  %fonttitle=\interfont\bfseries, 
  colback=white,
  colframe=white,
  boxrule=0pt,
  colbacktitle=myblue,
  coltitle=white,
  left=0pt, right=0pt, top=0pt, bottom=0pt, % content padding
  attach boxed title to top center={yshift=1mm}, % optional: title offset
  boxed title style={colback=myblue, boxrule=0pt, size=normal, boxsep=1mm} % padding for title
}}

\tcbset{
splitBox/.style={
    listing side text,
    colback=white,
    colframe=white,
    sharp corners,
    sidebyside,
    boxrule=1pt,
    skin=bicolor,
    colback=gray!10,
    colbacklower=white,
    valign=center,
    top=2pt, bottom=2pt, left=2pt, right=2pt
    %somehow we need to make the left side have a smaller right hand margin
}}

\tcbset{
infoBox/.style={
    colback={red!10},   % light peach background
    left=10mm, right=2mm, top=2mm, bottom=2mm, % padding
    boxrule=0pt,              % no full box border
    borderline west={2pt}{0pt}{red}, % only left border
    enhanced,                 % enable advanced options
    drop shadow,
    rounded corners,
    overlay={
    \node[anchor=north west, xshift=3mm, yshift=-2mm] at (frame.north west)
    {
    \textcolor{red}{\faInfoCircle}};
    }
}}